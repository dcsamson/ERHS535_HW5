% Options for packages loaded elsewhere
\PassOptionsToPackage{unicode}{hyperref}
\PassOptionsToPackage{hyphens}{url}
%
\documentclass[
]{article}
\usepackage{amsmath,amssymb}
\usepackage{iftex}
\ifPDFTeX
  \usepackage[T1]{fontenc}
  \usepackage[utf8]{inputenc}
  \usepackage{textcomp} % provide euro and other symbols
\else % if luatex or xetex
  \usepackage{unicode-math} % this also loads fontspec
  \defaultfontfeatures{Scale=MatchLowercase}
  \defaultfontfeatures[\rmfamily]{Ligatures=TeX,Scale=1}
\fi
\usepackage{lmodern}
\ifPDFTeX\else
  % xetex/luatex font selection
\fi
% Use upquote if available, for straight quotes in verbatim environments
\IfFileExists{upquote.sty}{\usepackage{upquote}}{}
\IfFileExists{microtype.sty}{% use microtype if available
  \usepackage[]{microtype}
  \UseMicrotypeSet[protrusion]{basicmath} % disable protrusion for tt fonts
}{}
\makeatletter
\@ifundefined{KOMAClassName}{% if non-KOMA class
  \IfFileExists{parskip.sty}{%
    \usepackage{parskip}
  }{% else
    \setlength{\parindent}{0pt}
    \setlength{\parskip}{6pt plus 2pt minus 1pt}}
}{% if KOMA class
  \KOMAoptions{parskip=half}}
\makeatother
\usepackage{xcolor}
\usepackage[margin=1in]{geometry}
\usepackage{color}
\usepackage{fancyvrb}
\newcommand{\VerbBar}{|}
\newcommand{\VERB}{\Verb[commandchars=\\\{\}]}
\DefineVerbatimEnvironment{Highlighting}{Verbatim}{commandchars=\\\{\}}
% Add ',fontsize=\small' for more characters per line
\usepackage{framed}
\definecolor{shadecolor}{RGB}{248,248,248}
\newenvironment{Shaded}{\begin{snugshade}}{\end{snugshade}}
\newcommand{\AlertTok}[1]{\textcolor[rgb]{0.94,0.16,0.16}{#1}}
\newcommand{\AnnotationTok}[1]{\textcolor[rgb]{0.56,0.35,0.01}{\textbf{\textit{#1}}}}
\newcommand{\AttributeTok}[1]{\textcolor[rgb]{0.13,0.29,0.53}{#1}}
\newcommand{\BaseNTok}[1]{\textcolor[rgb]{0.00,0.00,0.81}{#1}}
\newcommand{\BuiltInTok}[1]{#1}
\newcommand{\CharTok}[1]{\textcolor[rgb]{0.31,0.60,0.02}{#1}}
\newcommand{\CommentTok}[1]{\textcolor[rgb]{0.56,0.35,0.01}{\textit{#1}}}
\newcommand{\CommentVarTok}[1]{\textcolor[rgb]{0.56,0.35,0.01}{\textbf{\textit{#1}}}}
\newcommand{\ConstantTok}[1]{\textcolor[rgb]{0.56,0.35,0.01}{#1}}
\newcommand{\ControlFlowTok}[1]{\textcolor[rgb]{0.13,0.29,0.53}{\textbf{#1}}}
\newcommand{\DataTypeTok}[1]{\textcolor[rgb]{0.13,0.29,0.53}{#1}}
\newcommand{\DecValTok}[1]{\textcolor[rgb]{0.00,0.00,0.81}{#1}}
\newcommand{\DocumentationTok}[1]{\textcolor[rgb]{0.56,0.35,0.01}{\textbf{\textit{#1}}}}
\newcommand{\ErrorTok}[1]{\textcolor[rgb]{0.64,0.00,0.00}{\textbf{#1}}}
\newcommand{\ExtensionTok}[1]{#1}
\newcommand{\FloatTok}[1]{\textcolor[rgb]{0.00,0.00,0.81}{#1}}
\newcommand{\FunctionTok}[1]{\textcolor[rgb]{0.13,0.29,0.53}{\textbf{#1}}}
\newcommand{\ImportTok}[1]{#1}
\newcommand{\InformationTok}[1]{\textcolor[rgb]{0.56,0.35,0.01}{\textbf{\textit{#1}}}}
\newcommand{\KeywordTok}[1]{\textcolor[rgb]{0.13,0.29,0.53}{\textbf{#1}}}
\newcommand{\NormalTok}[1]{#1}
\newcommand{\OperatorTok}[1]{\textcolor[rgb]{0.81,0.36,0.00}{\textbf{#1}}}
\newcommand{\OtherTok}[1]{\textcolor[rgb]{0.56,0.35,0.01}{#1}}
\newcommand{\PreprocessorTok}[1]{\textcolor[rgb]{0.56,0.35,0.01}{\textit{#1}}}
\newcommand{\RegionMarkerTok}[1]{#1}
\newcommand{\SpecialCharTok}[1]{\textcolor[rgb]{0.81,0.36,0.00}{\textbf{#1}}}
\newcommand{\SpecialStringTok}[1]{\textcolor[rgb]{0.31,0.60,0.02}{#1}}
\newcommand{\StringTok}[1]{\textcolor[rgb]{0.31,0.60,0.02}{#1}}
\newcommand{\VariableTok}[1]{\textcolor[rgb]{0.00,0.00,0.00}{#1}}
\newcommand{\VerbatimStringTok}[1]{\textcolor[rgb]{0.31,0.60,0.02}{#1}}
\newcommand{\WarningTok}[1]{\textcolor[rgb]{0.56,0.35,0.01}{\textbf{\textit{#1}}}}
\usepackage{graphicx}
\makeatletter
\def\maxwidth{\ifdim\Gin@nat@width>\linewidth\linewidth\else\Gin@nat@width\fi}
\def\maxheight{\ifdim\Gin@nat@height>\textheight\textheight\else\Gin@nat@height\fi}
\makeatother
% Scale images if necessary, so that they will not overflow the page
% margins by default, and it is still possible to overwrite the defaults
% using explicit options in \includegraphics[width, height, ...]{}
\setkeys{Gin}{width=\maxwidth,height=\maxheight,keepaspectratio}
% Set default figure placement to htbp
\makeatletter
\def\fps@figure{htbp}
\makeatother
\setlength{\emergencystretch}{3em} % prevent overfull lines
\providecommand{\tightlist}{%
  \setlength{\itemsep}{0pt}\setlength{\parskip}{0pt}}
\setcounter{secnumdepth}{-\maxdimen} % remove section numbering
\ifLuaTeX
  \usepackage{selnolig}  % disable illegal ligatures
\fi
\usepackage{bookmark}
\IfFileExists{xurl.sty}{\usepackage{xurl}}{} % add URL line breaks if available
\urlstyle{same}
\hypersetup{
  pdftitle={Samson\_HW5},
  pdfauthor={Danielle Samson},
  hidelinks,
  pdfcreator={LaTeX via pandoc}}

\title{Samson\_HW5}
\author{Danielle Samson}
\date{2025-12-01}

\begin{document}
\maketitle

\subsection{Choose a city to work
with}\label{choose-a-city-to-work-with}

In the following chunk, I read in the data, looked through the list of
potential cities then decided to work with Denver. I then filtered the
dataset to include only the data from Denver and used the provided
coordinates to turn the dataframe into a simple feature object. Finally,
I generated the plot to help visualize where I am and where I need to go
with my code

\begin{Shaded}
\begin{Highlighting}[]
\FunctionTok{library}\NormalTok{(tidyverse)}
\FunctionTok{library}\NormalTok{(sf)}
\FunctionTok{library}\NormalTok{(tigris)}
\FunctionTok{library}\NormalTok{(ggthemes)}
\CommentTok{\#read in the data}

\NormalTok{homicides }\OtherTok{\textless{}{-}} \FunctionTok{read.csv}\NormalTok{(}\StringTok{"homicide{-}data.csv"}\NormalTok{)}

\CommentTok{\#choose a city}

\FunctionTok{print}\NormalTok{(}\FunctionTok{unique}\NormalTok{(homicides}\SpecialCharTok{$}\NormalTok{city))}
\end{Highlighting}
\end{Shaded}

\begin{verbatim}
##  [1] "Albuquerque"    "Atlanta"        "Baltimore"      "Baton Rouge"   
##  [5] "Birmingham"     "Boston"         "Buffalo"        "Charlotte"     
##  [9] "Chicago"        "Cincinnati"     "Columbus"       "Dallas"        
## [13] "Denver"         "Detroit"        "Durham"         "Fort Worth"    
## [17] "Fresno"         "Houston"        "Indianapolis"   "Jacksonville"  
## [21] "Kansas City"    "Las Vegas"      "Long Beach"     "Los Angeles"   
## [25] "Louisville"     "Memphis"        "Miami"          "Milwaukee"     
## [29] "Minneapolis"    "Nashville"      "New Orleans"    "New York"      
## [33] "Oakland"        "Oklahoma City"  "Omaha"          "Philadelphia"  
## [37] "Phoenix"        "Pittsburgh"     "Richmond"       "San Antonio"   
## [41] "Sacramento"     "Savannah"       "San Bernardino" "San Diego"     
## [45] "San Francisco"  "St. Louis"      "Stockton"       "Tampa"         
## [49] "Tulsa"          "Washington"
\end{verbatim}

\begin{Shaded}
\begin{Highlighting}[]
\NormalTok{denver\_homicides }\OtherTok{\textless{}{-}}\NormalTok{ homicides }\SpecialCharTok{\%\textgreater{}\%}
  \FunctionTok{filter}\NormalTok{(city }\SpecialCharTok{==} \StringTok{\textquotesingle{}Denver\textquotesingle{}}\NormalTok{) }\SpecialCharTok{\%\textgreater{}\%} 
  \FunctionTok{st\_as\_sf}\NormalTok{(}\AttributeTok{coords =} \FunctionTok{c}\NormalTok{(}\StringTok{"lon"}\NormalTok{, }\StringTok{"lat"}\NormalTok{),}
               \AttributeTok{crs =} \DecValTok{4269}\NormalTok{)}

\CommentTok{\#evaluate the current plot}

\NormalTok{denver\_homicides }\SpecialCharTok{\%\textgreater{}\%}
  \FunctionTok{ggplot}\NormalTok{() }\SpecialCharTok{+}
  \FunctionTok{geom\_sf}\NormalTok{()}
\end{Highlighting}
\end{Shaded}

\includegraphics{Samson_HW5_files/figure-latex/choose a city-1.pdf}

\subsection{Input city boundaries}\label{input-city-boundaries}

In this chunk, I took the Colorado geometry from the tigris package and
then filtered the data to include only Denver. Looking back, I think I
could have done this in one step by immediately filtering by the NAME
column. Again, I checked in on the plot

\begin{Shaded}
\begin{Highlighting}[]
\CommentTok{\#extract denver boundaries from tigris}
\NormalTok{places }\OtherTok{\textless{}{-}}\NormalTok{ tigris}\SpecialCharTok{::}\FunctionTok{places}\NormalTok{(}\AttributeTok{state =} \StringTok{"CO"}\NormalTok{,}
                                    \AttributeTok{cb =} \ConstantTok{FALSE}\NormalTok{,}
                                    \AttributeTok{year =} \StringTok{"2024"}\NormalTok{)}
\NormalTok{denver\_boundaries }\OtherTok{\textless{}{-}}\NormalTok{ places }\SpecialCharTok{\%\textgreater{}\%}
  \FunctionTok{filter}\NormalTok{(NAME }\SpecialCharTok{==} \StringTok{"Denver"}\NormalTok{)}
\end{Highlighting}
\end{Shaded}

\begin{Shaded}
\begin{Highlighting}[]
\NormalTok{denver\_boundaries }\SpecialCharTok{\%\textgreater{}\%}
  \FunctionTok{ggplot}\NormalTok{() }\SpecialCharTok{+}
  \FunctionTok{geom\_sf}\NormalTok{()}
\end{Highlighting}
\end{Shaded}

\includegraphics{Samson_HW5_files/figure-latex/evaluate the current plot-1.pdf}

\subsection{Align Geometries}\label{align-geometries}

In this chunk, I ensure that my geometries align before continuing to
manipulate my dataset

\begin{Shaded}
\begin{Highlighting}[]
\FunctionTok{ggplot}\NormalTok{() }\SpecialCharTok{+}
  \FunctionTok{geom\_sf}\NormalTok{(}\AttributeTok{data =}\NormalTok{ denver\_boundaries) }\SpecialCharTok{+}
  \FunctionTok{geom\_sf}\NormalTok{(}\AttributeTok{data =}\NormalTok{ denver\_homicides)}
\end{Highlighting}
\end{Shaded}

\includegraphics{Samson_HW5_files/figure-latex/ensure geometry alignment-1.pdf}

\subsection{Factor Lump to Most Murdered
Races}\label{factor-lump-to-most-murdered-races}

In this chunk, I convert the victim\_race column to a factor to be used
with the fct\_lump function to highlight the races with the most
murders. I then visualize the status of the plot

\begin{Shaded}
\begin{Highlighting}[]
\NormalTok{denver\_victim\_races }\OtherTok{\textless{}{-}}\NormalTok{ denver\_homicides }\SpecialCharTok{\%\textgreater{}\%}
  \FunctionTok{mutate}\NormalTok{(}\AttributeTok{victim\_race =} \FunctionTok{as.factor}\NormalTok{(victim\_race)) }\SpecialCharTok{\%\textgreater{}\%}
  \FunctionTok{mutate}\NormalTok{(}\AttributeTok{victim\_race =} \FunctionTok{fct\_lump}\NormalTok{(victim\_race, }\AttributeTok{n =} \DecValTok{3}\NormalTok{))}

\FunctionTok{ggplot}\NormalTok{() }\SpecialCharTok{+}
  \FunctionTok{geom\_sf}\NormalTok{(}\AttributeTok{data =}\NormalTok{ denver\_boundaries) }\SpecialCharTok{+}
  \FunctionTok{geom\_sf}\NormalTok{(}\AttributeTok{data =}\NormalTok{ denver\_victim\_races, }\FunctionTok{aes}\NormalTok{(}\AttributeTok{color =}\NormalTok{ victim\_race)) }
\end{Highlighting}
\end{Shaded}

\includegraphics{Samson_HW5_files/figure-latex/unnamed-chunk-2-1.pdf}

\subsection{Unsolved cases}\label{unsolved-cases}

In this chunk, I create a new column to sort the cases by solved vs
unsolved

\begin{Shaded}
\begin{Highlighting}[]
\CommentTok{\#solved vs unsolved columns}
\NormalTok{case\_status }\OtherTok{\textless{}{-}}\NormalTok{ denver\_victim\_races }\SpecialCharTok{\%\textgreater{}\%}
  \FunctionTok{mutate}\NormalTok{(}\AttributeTok{solved =} \FunctionTok{ifelse}\NormalTok{(disposition }\SpecialCharTok{\%in\%} \FunctionTok{c}\NormalTok{(}\StringTok{"Closed without arrest"}\NormalTok{, }\StringTok{"Closed by arrest"}\NormalTok{), }\StringTok{"Solved"}\NormalTok{, }\StringTok{"Unsolved"}\NormalTok{))}
\end{Highlighting}
\end{Shaded}

\subsection{Final Plot}\label{final-plot}

This chunk generates the final plot with maps separated by solved and
unsolved cases

\begin{Shaded}
\begin{Highlighting}[]
\FunctionTok{ggplot}\NormalTok{() }\SpecialCharTok{+}
  \FunctionTok{geom\_sf}\NormalTok{(}\AttributeTok{data =}\NormalTok{ denver\_boundaries) }\SpecialCharTok{+}
  \FunctionTok{geom\_sf}\NormalTok{(}\AttributeTok{data =}\NormalTok{ case\_status, }\FunctionTok{aes}\NormalTok{(}\AttributeTok{color =}\NormalTok{ victim\_race), }\AttributeTok{alpha =} \FloatTok{0.5}\NormalTok{) }\SpecialCharTok{+}
  \FunctionTok{facet\_wrap}\NormalTok{(}\StringTok{"solved"}\NormalTok{) }\SpecialCharTok{+}
  \FunctionTok{theme\_stata}\NormalTok{() }\SpecialCharTok{+} 
  \FunctionTok{labs}\NormalTok{(}
    \AttributeTok{title =} \StringTok{"Denver Homicides by Race"}\NormalTok{, }
    \AttributeTok{color =} \StringTok{"Victim Race"}
\NormalTok{  )}
\end{Highlighting}
\end{Shaded}

\includegraphics{Samson_HW5_files/figure-latex/unnamed-chunk-4-1.pdf}

\end{document}
